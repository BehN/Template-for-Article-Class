\documentclass{article}
\usepackage{siunitx}
\begin{document}
%-------------------------------------------------------------------------------
\section{Some Fancy Stuff} \label{app:fancy}
%-------------------------------------------------------------------------------
The \verb@siunitx.sty@ package greatly simplifies TeXing when writing scientific documents, where units and numbers are a big part of the writing. This package adds commands like:\\
\SI{10}{\henry} \\
\num{+-2,345},  \num{1ie10},  \num{+-i2,345e13} \\
\SI{10}{\Omega}, \SI{30}{\ohm}, 40~\si{\ohm}\\
\si{\kilogram\metre\per\second} \\
\si{\radian/\second}\\

\par \noindent Or, as shown in the following below.
\begin{singlespace}
\begin{center}
\begin{tabular}{|c|c|c|}
\hline
Name              & Notation            & \verb@\si@  \\ 
                  &                     & command\\ \hline
ampere            & \si{\ampere}        & \verb@\si{\ampere}@ \\ \hline 
centimeter        & \si{\centi\metre}   & \verb@\si{\centi\metre}@\\ \hline
Angstrom          & \si{\angstrom}      & \verb@\si{\angstrom}@\\ \hline  
degree Centigrade & \si{\degreeCelsius} & \verb@\si{\degreeCelsius}@\\ \hline
kelvin            & \si{\kelvin}        & \verb@\si{\kelvin}@\\\hline 
Henry             & \si{\henry}         & \verb@\si{\henry}@\\\hline  
Second            & \si{\second}        & \verb@\si{\second}@\\ \hline     
Volt:             & \si{\volt}          & \verb@\si{\volt}@\\ \hline        						
electron volt     & \si{\electronvolt}  & \verb@\si{\electronvolt}@\\ \hline
Farad             & \si{\farad}         & \verb@\si{\farad}@\\ \hline
meter             & \si{\meter}         & \verb@\si{\meter}@ \\ \hline
Ohm               & \si{\ohm}           & \verb@\si{\ohm}@ \\ \hline
\end{tabular}
\end{center}
\end{singlespace}
%-------------------------------------------------------------------------------
\end{document}